%%%%%%%%%%%%%%%%%
% This is an sample CV template created using altacv.cls
% (v1.1.3, 30 April 2017) written by LianTze Lim (liantze@gmail.com). Now compiles with pdfLaTeX, XeLaTeX and LuaLaTeX.
% 
%% It may be distributed and/or modified under the
%% conditions of the LaTeX Project Public License, either version 1.3
%% of this license or (at your option) any later version.
%% The latest version of this license is in
%%    http://www.latex-project.org/lppl.txt
%% and version 1.3 or later is part of all distributions of LaTeX
%% version 2003/12/01 or later.
%%%%%%%%%%%%%%%%

%% If you need to pass whatever options to xcolor
\PassOptionsToPackage{dvipsnames}{xcolor}

%% If you are using \orcid or academicons
%% icons, make sure you have the academicons 
%% option here, and compile with XeLaTeX
%% or LuaLaTeX.
% \documentclass[10pt,a4paper,academicons]{altacv}

%% Use the "normalphoto" option if you want a normal photo instead of cropped to a circle
% \documentclass[10pt,a4paper,normalphoto]{altacv}

\documentclass[10pt,a4paper]{altacv}
%% AltaCV uses the fontawesome and academicon fonts
%% and packages. 
%% See texdoc.net/pkg/fontawecome and http://texdoc.net/pkg/academicons for full list of symbols.
%% 
%% Compile with LuaLaTeX for best results. If you
%% want to use XeLaTeX, you may need to install
%% Academicons.ttf in your operating system's font 
%% folder.


% Change the page layout if you need to
\geometry{left=1cm,right=9cm,marginparwidth=6.8cm,marginparsep=1.2cm,top=1.25cm,bottom=1.25cm,footskip=2\baselineskip}

% Change the font if you want to.

% If using pdflatex:
\usepackage[T1]{fontenc}
\usepackage[utf8]{inputenc}
\usepackage[default]{lato}

% If using xelatex or lualatex:
% \setmainfont{Lato}

% Change the colours if you want to
\definecolor{Mulberry}{HTML}{008CFF}
\definecolor{MyBlue}{HTML}{008CFF}
\definecolor{SlateGrey}{HTML}{2E2E2E}
\definecolor{LightGrey}{HTML}{666666}
\colorlet{heading}{Black}
\colorlet{accent}{MyBlue}
\colorlet{emphasis}{SlateGrey}
\colorlet{body}{LightGrey}

% Change the bullets for itemize and rating marker
% for \cvskill if you want to
\renewcommand{\itemmarker}{{\small\textbullet}}
\renewcommand{\ratingmarker}{\faCircle}
%% sample.bib contains your publications
\addbibresource{sample.bib}

\usepackage[colorlinks]{hyperref}

\begin{document}

\name{José Guilherme Vanz }
\tagline{Software engineer \& lifelong learner}
%\photo{2.8cm}{Globe_High}
\personalinfo{%
  % Not all of these are required!
  % You can add your own with \printinfo{symbol}{detail}
  \email{jvanz@jvanz.com}
  \phone{+55 47 99919 3974}
  \location{Blumenau, Brasil}
  \homepage{jvanz.com}
  \linkedin{linkedin.com/in/jvanz}
  \github{github.com/jvanz}
  %% You MUST add the academicons option to \documentclass, then compile with LuaLaTeX or XeLaTeX, if you want to use \orcid or other academicons commands.
%   \orcid{orcid.org/0000-0000-0000-0000}
}

%% Make the header extend all the way to the right, if you want. 
\begin{fullwidth}
\makecvheader
\end{fullwidth}

%% Provide the file name containing the sidebar contents as an optional parameter to \cvsection.
%% You can always just use \marginpar{...} if you do
%% not need to align the top of the contents to any
%% \cvsection title in the "main" bar.
\cvsection[page1sidebar]{Experience}

\cvevent{Senior Software Engineer}{SUSE}{Sept 2021 -- Ongoing}{Nurnberg, Germany}
Developing R&D projects in the container and Kubernetes ecosystem making it thrive and turn into products. Currently, working on the security space in the Kubewarden project.
\begin{itemize}
\item Developed new features in the Kubewarden project.
\item Added tests to the whole Kubewarden stack.
\item Fixed bug reported by internal users and by the community.
\item Helped users with question and issues.

\end{itemize}
\cvevent{SUSE Container as a Service Platform Engineer - L3 Engineer}{SUSE}{Nov 2018 -- Aug 2021}{Nurnberg, Germany}
Working close to the engineering team to ensure the customers have a Kubernetes cluster up and running. Job focused in fixing bugs in any component of the SUSE Container as a Service (CaaS) Platform. Furthermore, as a L3 engineer, also help customers with any other SUSE product like SUSE Linux Enterprise Server, SUSE Enterprise Storage, SUSE Manager, SUSE Cloud among others.
\begin{itemize}
\item Troubleshoot issues in the SUSE CaaS components. Among them, Salt scripts, base operating systems, web interface, containers technologies and related tools.
\item Bug fixes in any component of SUSE products, but with focus on SUSE CaaS Platform.
\item Started to build an analysis tool based on the SUSE CaaS logs. Helping identifying patterns and known issues
\item Member of the internal tooling team writing the tool to deliver patches in applications running inside containers. 
\end{itemize}

\divider

\cvevent{Software Engineer}{ZPE Systems}{July 2015 -- Nov 2018}{Fremont, California, USA}
Working in a distributed team at Brazil, Ireland and USA, ZPE Systems helps IT staff reduce downtime by quickly and easily controlling IT devices
\begin{itemize}
\item Modernized Web App rewriting from scratch  a XSLT based application into a node.js application 
\item Enabled the platform to run Docker container engine
\item Brought overlayfs to an older Linux kernel version (only for tests)
\item C/C++, Javascript development
\end{itemize}

\cvsection{Volunteer}

%\cvevent{Software Engineer}{Apache Mesos, The Apache Software Foundation}{}{}
%\begin{itemize}
%\item Small code contributions in the distributed systems world.
%\item \href{https://github.com/apache/mesos/commits/master?author=jvanz}{apache/mesos}
%\end{itemize}


\cvevent{Maintainer}{Querido Diário, Open Knowledge Brasil }{}{}
\begin{itemize}
\item Build and maintain a software to find, analyze and monitor public spend from Brazilian city halls. Repositories: \href{https://github.com/okfn-brasil/querido-diario}{querido-diario}, \href{https://github.com/okfn-brasil/querido-diario-api}{querido-diario-api}
,\href{https://github.com/okfn-brasil/querido-diario-data-processing}{querido-diario-data-processing}, \href{https://github.com/okfn-brasil/querido-diario-toolbox/}{querido-diario-toolbox}

\end{itemize}


\cvevent{Software Engineer}{ScyllaDB}{}{}
\begin{itemize}
\item Small code contributions in the NoSQL database world and high performance computing (Seastar framework).
\item \href{https://github.com/scylladb/scylla/}{scylladb/scylla} and \href{https://github.com/scylladb/seastar}{scylladb/seastar}
\end{itemize}

\cvevent{Software Engineer}{Container technologies}{}{}
\begin{itemize}
\item Small code contributions to container tools.
\item \href{https://github.com/containers/image/commits?author=jvanz}{containers/image},  \href{https://github.com/containers/skopeo/commits?author=jvanz}{containers/skopeo} and \href{https://github.com/containers/libpod/commits?author=jvanz}{containers/libpod}
\end{itemize}

%\cvevent{Co-founder}{Hackerspace Blumenau}{}{}
%\begin{itemize}
%\item Organize technologies meetups promoting the knowledge sharing, specially in the Open Source world
%\end{itemize}

\clearpage

\end{document}
