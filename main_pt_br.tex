%%%%%%%%%%%%%%%%%
% This is an sample CV template created using altacv.cls
% (v1.1.3, 30 April 2017) written by LianTze Lim (liantze@gmail.com). Now compiles with pdfLaTeX, XeLaTeX and LuaLaTeX.
% 
%% It may be distributed and/or modified under the
%% conditions of the LaTeX Project Public License, either version 1.3
%% of this license or (at your option) any later version.
%% The latest version of this license is in
%%    http://www.latex-project.org/lppl.txt
%% and version 1.3 or later is part of all distributions of LaTeX
%% version 2003/12/01 or later.
%%%%%%%%%%%%%%%%

%% If you need to pass whatever options to xcolor
\PassOptionsToPackage{dvipsnames}{xcolor}

%% If you are using \orcid or academicons
%% icons, make sure you have the academicons 
%% option here, and compile with XeLaTeX
%% or LuaLaTeX.
% \documentclass[10pt,a4paper,academicons]{altacv}

%% Use the "normalphoto" option if you want a normal photo instead of cropped to a circle
% \documentclass[10pt,a4paper,normalphoto]{altacv}

\documentclass[10pt,a4paper]{altacv}
%% AltaCV uses the fontawesome and academicon fonts
%% and packages. 
%% See texdoc.net/pkg/fontawecome and http://texdoc.net/pkg/academicons for full list of symbols.
%% 
%% Compile with LuaLaTeX for best results. If you
%% want to use XeLaTeX, you may need to install
%% Academicons.ttf in your operating system's font 
%% folder.


% Change the page layout if you need to
\geometry{left=1cm,right=9cm,marginparwidth=6.8cm,marginparsep=1.2cm,top=1.25cm,bottom=1.25cm,footskip=2\baselineskip}

% Change the font if you want to.

% If using pdflatex:
\usepackage[T1]{fontenc}
\usepackage[utf8]{inputenc}
\usepackage[default]{lato}

% If using xelatex or lualatex:
% \setmainfont{Lato}

% Change the colours if you want to
\definecolor{Mulberry}{HTML}{008CFF}
\definecolor{MyBlue}{HTML}{008CFF}
\definecolor{SlateGrey}{HTML}{2E2E2E}
\definecolor{LightGrey}{HTML}{666666}
\colorlet{heading}{Black}
\colorlet{accent}{MyBlue}
\colorlet{emphasis}{SlateGrey}
\colorlet{body}{LightGrey}

% Change the bullets for itemize and rating marker
% for \cvskill if you want to
\renewcommand{\itemmarker}{{\small\textbullet}}
\renewcommand{\ratingmarker}{\faCircle}
%% sample.bib contains your publications
\addbibresource{sample.bib}

\usepackage[colorlinks]{hyperref}

\begin{document}

\name{José Guilherme Vanz }
\tagline{Engenheiro de Software e eterno aprendiz}
%\photo{2.8cm}{Globe_High}
\personalinfo{%
  % Not all of these are required!
  % You can add your own with \printinfo{symbol}{detail}
  \email{my@email.com}
  \phone{+<phone>}
  \location{<location>}
  \homepage{<website>}
  \linkedin{linkedin.com/in/<user>}
  \github{github.com/<user>}
  %% You MUST add the academicons option to \documentclass, then compile with LuaLaTeX or XeLaTeX, if you want to use \orcid or other academicons commands.
%   \orcid{orcid.org/0000-0000-0000-0000}
}

%% Make the header extend all the way to the right, if you want. 
\begin{fullwidth}
\makecvheader
\end{fullwidth}

%% Provide the file name containing the sidebar contents as an optional parameter to \cvsection.
%% You can always just use \marginpar{...} if you do
%% not need to align the top of the contents to any
%% \cvsection title in the "main" bar.
\cvsection[page1sidebar_pt_br]{Experinciência}

\cvevent{Engenheiro de Software}{ZPE Systems}{Julho 2015 -- Atualmente}{Fremont, California, EUA}
Trabalhando em um time distribuído entre Brasil, Irlanda e EUA, ZPE Systems ajuda os profissionais de TI a reduzir o tempo offline ajudando a realizar um controle rápido e fácil de todos os aparelhos de TI.
\begin{itemize}
\item Modernizou a stack web do sistema.Convertendo uma aplicação baseada em XSLT para node.js 
\item Fez ser possível rodar Docker na distribuição Linux da ZPE Systems
\item Aplicou pacthes no Kernel Linux para permitir rodar overlayfs em um Kernel mais antigo. Feito apenas para teste
\item Desenvolvimento C/C++ e Javascript
\item Trabalho em equipe distribuída entre Irlanda, Brasil e EUA
\end{itemize}

\divider

\cvevent{Engenheiro de Software}{Lemonade}{Junho 2013 -- Julho 2015}{Blumenau, Santa Catarina, Brazil}
Usando a tecnologia para criar momentos marcas e clientes
\begin{itemize}
\item Contriui aplicativos Android para experiência de marca
\item Desenvolvimento de parte de um motor gráfico em C++ no Android (NDK) integrando com a parte Java do aplicativo.
\item Contruiu do zero uma plataforma para medir o engajamento em campanhas de marca dentro do aplicativo desenvolvido pela empresa
\item Desenvolvimento Web com Python.
\end{itemize}

\cvsection{Voluntário}

\cvevent{Engenheiro de Software}{Apache Mesos, The Apache Software Foundation}{}{}
\begin{itemize}
\item Pequenas contribuições de código no mundo dos sistemas distribuídos.
\item \href{https://goo.gl/etn1tL}{https://goo.gl/etn1tL}
\end{itemize}


\cvevent{Engenheiro de Software}{Libreoffice, The Document Foundation}{}{}
\begin{itemize}
\item \href{https://goo.gl/bckJKL}{https://goo.gl/bckJKL}
\end{itemize}

\cvevent{Engenheiro de Software}{ScyllaDB}{}{}
\begin{itemize}
\item Pequenas contribuições de código no mundo de base de dados NoSQL e computação de alto desempenho (Seastar framework).
\item \href{https://goo.gl/EmnJvi}{https://goo.gl/EmnJvi}
\item \href{https://goo.gl/QTKU9z}{https://goo.gl/QTKU9z}
\end{itemize}


\cvevent{Organizador}{Hackerspace Blumenau}{}{}
\begin{itemize}
\item Organização de encontros para compartilhamento de conhecimento voltado para area de tecnologia e Open Source
\end{itemize}

\clearpage

\end{document}
